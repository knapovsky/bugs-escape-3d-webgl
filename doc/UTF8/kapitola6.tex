\chapter{Závěr}
\label{chap:zaver}
V rámci této bakalářské práce byla analyzována hra Berušky 2, na jejímž základě byl následně vytvořen návrh a implementace webové hry Berušky 2 WebGL. K vývoji bylo využito moderních webových technologií, které byly stručně představeny a uvedeny do souvislosti s jejich použitím. Veškerá funkčnost implementovaného řešení byla popsána pomocí jazyků JavaScript a GLSL pro popis shaderů grafické karty. Implementované řešení bylo podrobeno testům, jejichž výsledky byly diskutovány s konzultantem této práce - Ing. Martinem Stránským. Je důležité zdůraznit, že řešení rozšiřuje zadání práce o různé možnosti nastavení vykreslování, implementaci webu, přehrávání herní hudby a mnohá vylepšení v podobě animací, které vylepšují celkový dojem z celé aplikace.

Po konzultaci s panem Ing. Martinem Stránským byly použité technologie celkově zhodnoceny jako vhodné pro vývoj aplikací jako je ta, jejíž implementace byla součástí této práce. Pro komerční využití je však potřeba, aby byli nejprve odstraněny některé problémy, které jsou spojené s tím, že web a jeho prostředky nebyly pro tyto účely vytvořeny. WebGL například používá hardwarové akcelerace vykreslování, avšak aplikace využívající toto rozhraní stále nedosahují rychlosti aplikací desktopových. To, co podle provedených testů WebGL \textit{brzdí}, je výpočetní rychlost JavaScriptu. JavaScript není určen pro operace s velkým množstvím dat. Interprety tohoto jazyka jsou však dnes středem pozornosti a lze tedy předpokládat, že tento problém bude postupně mizet.

Existuje velké množství technik sloužících k zobrazení realisticky vypadající herní scény. Tato hra některé z nich implementuje, avšak stále zbývá mnoho prostoru pro optimalizace a vylepšení. Jednou z optimalizací, se kterou se hra v brzké době setká, je tzv. frustrum culling, pro nějž je hra již částečně připravena. Další je redukce přenášených dat. Pokud by to technologie umožnily, pak by bylo pro hráče jistě přívětivé, kdyby si mohl hru spustit z lokálního úložiště a nebyl tak závislý na datovém připojení. Pokud by se technologie vyhovující tomuto účelu objevila, pak bude její využití jednou z priorit. Při konzultacích s panem Stránským byla také navržena mnohá vylepšení v tom, jakým způsobem je hra ovládána. Tato vylepšení s ním budou dále diskutována.
