\chapter{Úvod}
V dnešní době již webové technologie pokročily natolik, že je možné vytvářet takový obsah, který je plnohodnotnou aplikací s uživatelským rozhraním, multimediálním obsahem, či možností komunikace jejích uživatelů. Největší výhodou těchto aplikací je však to, že jsou dostupné komukoliv s kompatibilním webovým prohlížečem. Multiplatformnost aplikací tak není řešena tím, že bychom vytvářeli pro každý typ operačního systému příslušné spustitelné soubory, ale stačí pouze otevřít prohlížeč, zadat adresu a následně s aplikací pracovat. 
%Hra, jejíž vývoj je v této práci popisován tyto technologie využívá a je tedy možné ji používat na rozličných typech zařízení včetně moderních telefonů a tabletů.

Cílem této práce bylo navrhnout a implementovat logickou webovou hru založenou na hře Berušky 2 a ověřit tak vhodnost použití nově se objevujících technologií k tomutu účelu.

Technologie, jejich historie a použití jsou popsány v kapitole~\ref{chap:teorie}. Ta se primárně zaměřuje na WebGL, avšak jsou zde obsaženy i informace ohledně technologie HTML5, programovacího jazyka JavaScript a reprezentaci dat ve formátu JSON. V této kapitole je také obsažena teorie nutná k pochopení principů použitých při implementaci hry. Kapitola~\ref{chap:analysis} se zabývá analýzou původní hry Berušky 2, na jejímž základě je následně vypracován návrh implementované hry. Návrh webu, pomocí něhož je hra dostupná, je zmíněn pouze okrajově, jelikož není přímou součástí této práce. Implementační témata jsou obsažena v kapitole~\ref{chap:implementace} a testy výsledného řešení a jeho porovnání s původní hrou v kapitole~\ref{chap:testing}. Kapitola~\ref{chap:zaver} následně shrnuje poznatky získané při vývoji a diskutuje použitelnost technologií.
